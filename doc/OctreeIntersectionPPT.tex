\documentclass[UTF8]{ctexbeamer}	% Compile at least twice!
%\setbeamertemplate{navigation symbols}{}
\usetheme{Madrid}
% \setbeamertemplate{navigation symbols}{}
% \useinnertheme{rectangles}
% \useoutertheme{infolines}
% \useoutertheme[title,section,subsection=true]{smoothbars}
\useoutertheme{split}
\useinnertheme{rounded}
\setbeamertemplate{headline}{}
\usecolortheme{beaver}


% \usecolortheme{default}
% \usecolortheme{whale}
 
% -------------------
% Packages
% -------------------
\usepackage{
    amsmath,			% Math Environments
    amssymb,			% Extended Symbols
    enumerate,		    % Enumerate Environments
    graphicx,			% Include Images
    lastpage,			% Reference Lastpage
    multicol,			% Use Multi-columns
    multirow,			% Use Multi-rows
    pifont,			    % For Checkmarks
    stmaryrd,            % For brackets
    listings,
    subfigure,
}
\usepackage[english]{babel}
\usepackage{graphicx}
\usepackage{animate}
\usepackage{xeCJK}
\usepackage{fontspec} 
% \setsansfont{Times New Roman}
% \setmonofont{Times New Roman}
\setmainfont{Times New Roman Cyr}
\setCJKmainfont{方正新楷体简体}

\newfontfamily\os{Open Sans}
\newfontfamily\oscl{Open Sans Condensed}
\newfontfamily\tnr{Times New Roman}
\newfontfamily\tnrc{Times New Roman Cyr}
\newfontfamily\tim{Times}
\newfontfamily\roc{Rockwell}
% \usepackage{CJK}
% \lstset{language=C++}
% \lstset{extendedchars=false}
% \lstset{breaklines}


% -------------------
% Colors
% -------------------
% \definecolor{UniOrange}{RGB}{212,69,0}
% \definecolor{UniGray}{RGB}{62,61,60}
% \definecolor{UniRed}{HTML}{B31B1B}
% \definecolor{UniGray}{HTML}{222222}
% \setbeamercolor{title}{fg=UniGray}
% \setbeamercolor{frametitle}{fg=UniOrange}
% \setbeamercolor{structure}{fg=UniOrange}
% \setbeamercolor{section in head/foot}{bg=UniGray}
% \setbeamercolor{author in head/foot}{bg=UniGray}
% \setbeamercolor{date in head/foot}{fg=UniGray}
% \setbeamercolor{structure}{fg=UniOrange}
% \setbeamercolor{local structure}{fg=black}
% \beamersetuncovermixins{\opaqueness<1>{0}}{\opaqueness<2->{15}}


% -------------------
% Fonts & Layout
% -------------------
% \usepackage{palatino}
\usefonttheme{serif}
% \setbeamerfont{title like}{shape=\scshape}
% \setbeamerfont{frametitle}{shape=\scshape}
% \setbeamertemplate{itemize items}[circle]
% \setbeamertemplate{enumerate items}[default]


% -------------------
% Commands
% -------------------

% Special Characters
% \newcommand{\N}{\mathbb{N}}
% \newcommand{\Z}{\mathbb{Z}}
% \newcommand{\Q}{\mathbb{Q}}
% \newcommand{\R}{\mathbb{R}}
%\newcommand{\C}{\mathbb{C}}

% Math Operators
% \DeclareMathOperator{\im}{im}
% \DeclareMathOperator{\Span}{span}

% Special Commands
% \newcommand{\pf}{\noindent\emph{Proof. }}
% \newcommand{\ds}{\displaystyle}
% \newcommand{\defeq}{\stackrel{\text{def}}{=}}
% \newcommand{\ov}[1]{\overline{#1}}
% \newcommand{\ma}[1]{\stackrel{#1}{\longrightarrow}}
% \newcommand{\twomatrix}[4]{\begin{pmatrix} #1 & #2 \ #3 & #4 \end{pmatrix}}


% -------------------
% Tikz & PGF
% -------------------
\usepackage{tikz}
\usepackage{tikz-cd}
\usetikzlibrary{
    calc,
    decorations.pathmorphing,
    matrix,arrows,
    positioning,
    shapes.geometric
}
\usepackage{pgfplots}
\pgfplotsset{compat=newest}
\usepackage{wrapfig}
\usepackage{cite}


% -------------------
% Theorem Environments
% -------------------
\usepackage{amsthm}
\theoremstyle{plain}
\newtheorem{sit}{Situation}[section]
\newtheorem{prop}{Proposition}[section]
\newtheorem{rtm}{Theorem}[section]
\newtheorem{cor}{Corollary}[section]
\theoremstyle{definition}
\newtheorem{das}{Data structure}[section]
\newtheorem{nex}{Non-Example}[section]
\newtheorem{cla}{class}[section]
\newtheorem{emt}{}[section]
\newtheorem{defn}{Definition}[section]
\theoremstyle{remark}
\newtheorem{rem}{Remark}[section] 
\numberwithin{equation}{section}

\newcommand\caesura{$\mkern -8.5mu\raise -.2ex\hbox{\rotatebox[]{180}{\`}}\ $}
\setbeamertemplate{caption}[numbered]
% -------------------
% Title Page
% -------------------
\title{\textcolor{red}{自适应的空间划分法加速三角形求交}}
%\subtitle{\textcolor{white}{Mathematics Conference for the Mysterious and dMagical}}  
\author{报告人: 谭焱 
% \newline \newline
%  \small{硕士导师: 王何宇\, \
%  申请博士导师: 张庆海}
 }

\institute{\small{小组成员: 邱云昊,谭焱}}
\date{\today} 


% -------------------
% Content
% -------------------
\begin{document}

% Title Page
\begin{frame}
  \titlepage
\end{frame}


\begin{frame}
  \frametitle{提纲}
  \tableofcontents
\end{frame}

\section{解决的问题}

\begin{frame}[fragile]
  \frametitle{研究背景}
  \begin{enumerate}
    \item 使用殷集对三维空间中有物理意义的区域建模.
    \item 为殷集实现布尔代数,提供研究流相拓扑变化的工具.
    \item 实现布尔运算时需要计算空间三角形的交线.
    \item 已有求交算法将所有三角形两两求交,
          \begin{itemize}
            \item 时间复杂度是$O(N^2)$.
            \item 是当前布尔运算的时间瓶颈(2112个三角形时占总时长
                  0.20 / 0.52, 359424个三角形时为115.39 / 140.36).
            \item 次要时间瓶颈是三角剖分(分别占用时间0.26 / 0.52, 19.65 / 140.36).
          \end{itemize}
    \item 需要高效的大量空间三角形求交方法,有效的提高布尔运算速度.
  \end{enumerate}
  \begin{figure}[htb]
    \centering
    \subfigure{\includegraphics[width=0.4\textwidth]{fig/Octree_s1p1.png}} \qquad
    \subfigure{\includegraphics[width=0.4\textwidth]{fig/Octree_s1p2.png}}
  \end{figure}
\end{frame}

\section{采用的方法}

\begin{frame}[fragile]
  \frametitle{空间划分}
  \begin{enumerate}
    \item 三角形相交是局部的,考虑局部区域中的三角形两两求交.如图所示
          \begin{figure}[htb]
            \centering
            \subfigure{\includegraphics[width=0.4\textwidth]{fig/Octree_s2p1.png}}
          \end{figure}
    \item 不考虑一个殷集边界上的三角形相交时,大部分区域可以忽略.
          \begin{figure}[htb]
            \centering
            \subfigure{\includegraphics[width=0.4\textwidth]{fig/Oc_s2p2.png}}
          \end{figure}
  \end{enumerate}
\end{frame}

\begin{frame}[fragile]
  \frametitle{自适应的空间划分}
  \begin{enumerate}
    \item 在三角形密集的局部进行更细致的划分,结合之前说的剪枝算法.
          \begin{figure}[htb]
            \centering
            \subfigure{\includegraphics[width=0.4\textwidth]{fig/Oc_s2p3.png}}
          \end{figure}
    \item 得到的自适应空间划分如下
          \begin{figure}[htb]
            \centering
            \subfigure{\includegraphics[width=0.4\textwidth]{fig/Oc_s2p4.png}}
          \end{figure}
  \end{enumerate}
\end{frame}


\section{编程实现}

\begin{frame}[fragile]
  \frametitle{代码结构}
  \begin{enumerate}
    \item 类接口
          \begin{itemize}
            \item 输入: 两组空间三角形, 空间划分树的最大深度.
            \item 输出: 每个三角形上的所有交线段.
          \end{itemize}
    \item 建立空间划分树的节点
          \begin{lstlisting}[language={[ANSI]C},numbers=left,
      numberstyle=\tiny,keywordstyle=\color{blue!70},commentstyle=\color{red!50!green!50!blue!50},frame=shadowbox,
      rulesepcolor=\color{red!20!green!20!blue!20}]
      template <class T>
      struct OctreeNode {
        T val;
        std::vector<int> tris[2];
        std::vector<OctreeNode<T>*> child;
      };
      
    \end{lstlisting}
    \item initOctree() 生成八叉树,每个节点的子节点是长方体的8等分之一.
    \item pruneTree() 对树进行剪枝,最小化三角形求交计算的次数.
    \item calTest() 遍历八叉树,计算所有叶节点长方体中的三角形求交.
  \end{enumerate}
\end{frame}

\begin{frame}[fragile]
  \frametitle{时间复杂度分析}
  \begin{enumerate}
    \item 设两个殷集边界分别约有$N$个三角形.
    \item 令八叉树深度为 $log_8N$, 可理解为八叉树将计算空间$N$等分.
    \item 不妨设殷集的三角形不会过大,即覆盖的长方体数量为常数.
    \item initOctree() 的时间复杂度为$O(NlogN)$.
    \item pruneTree() 的时间复杂度为$O(N)$.
    \item calTest() 最坏情况时间复杂度为$O(N^2)$,但在普遍情况是$O(N)$.
    \item 综上,该算法时间复杂度为$O(NlogN)$.
  \end{enumerate}
  \begin{figure}[htb]
    \centering
    \setcounter{subfigure}{0}
    \subfigure[(960, 1152)]{\includegraphics[width=0.3\textwidth]{fig/Oc_s2p5.png}}
    \subfigure[(16128, 18432)]{\includegraphics[width=0.3\textwidth]{fig/Oc_s2p6.png}}
    \subfigure[(261120, 98304)]{\includegraphics[width=0.3\textwidth]{fig/Oc_s2p7.png}}
    \caption{边界上(x1, x2)个三角形的殷集求交}
  \end{figure}
\end{frame}

\section{测试与问题}

\begin{frame}
  \frametitle{计算时间测试}
  \begin{enumerate}
    \item 自适应的空间划分法和已有方法计算时间比较如下
          \begin{figure}[htb]
            \centering
            \subfigure{\includegraphics[width=0.4\textwidth]{fig/Oc_s3p1.png}}
          \end{figure}
          \begin{table}[]
            \resizebox{.8\textwidth}{!}{%
              \begin{tabular}{|c|c|c|c|c|c|c|}
                \hline
                三角形个数/个 & 2112     & 10880    & 34560    & 114176    & 359424     & 539392     \\ \hline
                空间划分法/s  & 0.302020 & 1.549209 & 3.878483 & 9.649845  & 25.118267  & 35.477708  \\ \hline
                两两求交法/s  & 0.175366 & 1.140628 & 4.671456 & 22.934571 & 129.951804 & 306.352480 \\ \hline
              \end{tabular}%
            }
            \caption{新老方法计算时间对比}
            \label{tab:my-table}
          \end{table}
    \item 在当前测试图形中三角剖分消耗时间略小于空间划分法求交.
  \end{enumerate}
\end{frame}

\begin{frame}
  \frametitle{待解决的问题}
  \begin{enumerate}
    \item 程序没有通过所有测试.
    \item 判断长方体包含三角形的程序实现低效.
    \item 测试样例简单单一.
    \item 时间复杂度分析不严谨.
    \item 是否要在殷集内部添加可以表示三角形相邻关系的数据结构?
    \item 是否要继续优化?

  \end{enumerate}
\end{frame}


% % Motivation
% \section{个人基本情况介绍}

% % Definitions & Examples
% \begin{frame}[fragile]
%     \frametitle{学习情况}
%     \begin{enumerate}
%         \item 研究生课程
%               \begin{itemize}
%                   \item 科研相关课程获得良好成绩
%                         \begin{itemize}
%                             \item 非线性问题的数学方法(92), 图形学的新进展(90) 等.
%                         \end{itemize}
%                   \item 英语阅读及写作方面
%                         \begin{itemize}
%                             \item 六级489分(阅读205)可以流畅阅读英文文献.
%                             \item 通过研究生论文写作指导(92)打下坚实写作基础.
%                         \end{itemize}
%               \end{itemize}
%         \item 科研训练
%               \begin{itemize}
%                   \item 系统地学习了代数拓扑和微分方程数值解的相关理论.
%                   \item 锻炼了算法设计和编程实现能力.
%                   \item 培养了将代数拓扑和微分方程数值解结合进行科学研究的思维方式.
%               \end{itemize}
%     \end{enumerate}
% \end{frame}

% \section{硕士阶段的科研工作}
% \subsection{涉密的军工项目}
% \begin{frame}
%     \frametitle{涉密军工项目}
%     \begin{itemize}
%         \item 潜艇湍流尾迹的海洋表面特征等内波现象的研究, 用于潜艇追踪和隐身
%               (\textcolor{red}{军科委基础加强重点项目}).
%               \begin{figure}
%                   \includegraphics[width = 0.7\textwidth]{fig/s12.png}
%                   \caption{水下潜艇产生的各种尾迹示意图, 包括开尔文尾迹, 内波, 湍流尾迹, 涡尾迹, 煎
%                       饼旋涡.}
%               \end{figure}
%     \end{itemize}
% \end{frame}

% \subsection{\textcolor{red}{非涉密的三维殷集和布尔代数}}
% \begin{frame}
%     \frametitle{殷集的研究背景}
%     \begin{itemize}
%         \item 含有动边界的不可压流体在军工领域是非常重要的研究课题.
%         \item 现有方法是对界面的几何和拓扑问题进行回避,导致了:
%               \begin{enumerate}
%                   \item 对等距变换的流场不能保证几何性质.
%                   \item 对同胚映射的流场不能保证拓扑性质.
%                   \item 精度最高为二阶精度.
%                   \item 很难对拓扑变化进行严格的处理.
%               \end{enumerate}
%         \item 我们的核心思想是用\textcolor{red}{几何和拓扑的手段
%                   研究几何和拓扑的问题},其中首要工作在于对流相建模
%               (Zhang and Li, \tnr{ Math. Comput.}\tnrc{, \ 2020).}

%               %   \os (Zhang and Li 2020 Math. Comput.).

%               %   \oscl (Zhang and Li 2020 Math. Comput.).

%               %   \tnr (Zhang and Li 2020 Math. Comput.).

%               %   \tim (Zhang and Li 2020 Math. Comput.).

%               %   \roc (Zhang and Li 2020 Math. Comput.).
%     \end{itemize}
%     \begin{center}
%         \includegraphics[width = 0.9\textwidth]{fig/s14.png}
%     \end{center}
% \end{frame}


% \begin{frame}
%     \frametitle{殷集和布尔代数的重要性}
%     \begin{itemize}
%         \item 几何和拓扑的手段集中体现在殷空间和布尔代数.
%         \item 为连续介质提供了简单高效的表示(以O(1)时间得到欧拉示性数).
%         \item 给动边界的界面追踪的高保真算法提供理论支撑.
%         \item 给多相流的流相拓扑变化刻画和高阶算法设计提供理论支撑.
%         \item 是许多重大科学问题(军工项目)的核心难点.
%     \end{itemize}

%     \begin{columns}
%         \column{0.5\linewidth}<1->
%         \centering
%         \includegraphics[width = .6\textwidth]{fig/s18.png}
%         \column{0.5\linewidth}<1->
%         \centering
%         \includegraphics[width = .8\textwidth]{fig/s19.png}
%     \end{columns}
% \end{frame}


% \begin{frame}
%     \frametitle{二维殷集}
%     \begin{itemize}
%         \item 二维空间中,任一个殷集可以唯一表示为
%               \[\mathcal{Y} = \cup_j^{\bot \bot}\cap_i \text{int}(\gamma_{j, i} ),\]
%               约当曲线 $\gamma_{j, i}$是$\mathcal{Y}$内第j$个连通分量
%               的第i$条边界.
%         \item 高效实现了殷集上的布尔代数.
%     \end{itemize}
%     \begin{figure}[htb]
%         \centering
%         \subfigure{\includegraphics[width=0.3\textwidth]{fig/p.png}}
%         \subfigure{\includegraphics[width=0.3\textwidth]{fig/m.png}}
%         \subfigure{\includegraphics[width=0.3\textwidth]{fig/pm.png}}
%     \end{figure}
% \end{frame}

% \begin{frame}
%     \frametitle{科研成果---三维殷集的数学模型及布尔代数}
%     \begin{itemize}
%         \item
%               \textcolor{red}{二流形的分类定理} \newline
%               有向的紧二流形是同胚于球或者圆环或它们的有限个连通和.
%               \begin{center}
%                   \includegraphics[width = .9\textwidth]{fig/s15.png}
%               \end{center}


%         \item 黏合紧曲面是一个二维连通紧流形或这种流形的商空间, 其商映射
%               将多个与一维 CW 复形同胚的子集粘在一起; 将这个一维子集删除后
%               该黏合紧曲面仍然是连通的.
%     \end{itemize}
% \end{frame}

% \begin{frame}
%     \frametitle{三维殷集的唯一表示}
%     \begin{itemize}
%         \item \textcolor{red}{三维殷集}:三维空间中边界有界的正则半解析开集.所有三维殷集构
%               成的集合被称为殷空间,记为 $\mathbb{Y}$.
%         \item 任一个殷集$\mathcal{Y} \in \mathbb{Y}$可以唯一表示为
%               \[\mathcal{Y} = \cup_j^{\bot \bot} \cap_i \text{int}(\Gamma_{j, i}),\]
%               黏合紧曲面$\Gamma_{j, i}$是$\mathcal{Y}$的第j$个连通分量的第i$个边界.
%     \end{itemize}
%     \begin{figure}[htbp]
%         \centering
%         \subfigure[殷集$\mathcal{Y}$]{\includegraphics[width=0.4\textwidth]{fig/s1.png}} \qquad
%         \subfigure[表示$\mathcal{Y}$的两个黏合紧曲面]{\includegraphics[width=0.4\textwidth]{fig/s2.png}}
%         \caption{殷集和表示殷集的黏合紧曲面}
%         \vspace{0.2in}
%     \end{figure}
% \end{frame}

% \begin{frame}
%     \frametitle{布尔代数的实现方式}
%     \begin{enumerate}
%         \item 计算殷集边界上的所有非流形点.
%         \item 沿非流形点剪开黏合紧曲面得到若干曲面片.
%         \item 根据交并补的需要删除曲面片或改变曲面片方向.
%         \item 将曲面片重新黏合成黏合紧曲面集合.
%         \item 黏合紧曲面集合唯一表示一个三维殷集作为布尔运算结果.
%     \end{enumerate}
%     \begin{figure}[!htb]
%         \centering
%         \subfigure[殷集$\mathcal{Y}_1$]{\includegraphics[width=0.365\textwidth]{fig/s3.png}} \qquad
%         \subfigure[殷集$\mathcal{Y}_2$]{\includegraphics[width=0.4\textwidth]{fig/s4.png}}
%         \caption{将要进行布尔运算的殷集$\mathcal{Y}_1$和$\mathcal{Y}_2$}
%         \vspace{0.2in}
%     \end{figure}
% \end{frame}

% \begin{frame}
%     \frametitle{布尔运算结果图}
%     \begin{figure}[!htb]
%         \centering
%         \subfigure[$\partial\mathcal{Y}_1$剪开得到的部分曲面片]
%         {\includegraphics[width=0.36\textwidth]{fig/s16.png}} \qquad
%         \subfigure[$\partial\mathcal{Y}_2$剪开得到的曲面片]
%         {\includegraphics[width=0.375\textwidth]{fig/s17.png}}
%         %   \caption{sfd}
%         \vspace{-0.2in}
%     \end{figure}
%     \begin{figure}[!htb]
%         \centering
%         \subfigure[求交得到的殷集$\mathcal{Y}_3$]
%         {\includegraphics[width=0.35\textwidth]{fig/s5.png}} \qquad
%         \subfigure[构成$\partial\mathcal{Y}_3$的两个黏合紧曲面]
%         {\includegraphics[width=0.365\textwidth]{fig/s6.png}}
%         %   \caption{sfd}
%         %   \vspace{0.2in}
%     \end{figure}
% \end{frame}

% \begin{frame}
%     \begin{figure}[!htb]
%         \centering
%         \subfigure[求并得到的殷集$\mathcal{Y}_4$]
%         {\includegraphics[width=0.35\textwidth]{fig/s7.png}} \qquad
%         \subfigure[构成$\partial\mathcal{Y}_4$的两个黏合紧曲面]
%         {\includegraphics[width=0.365\textwidth]{fig/s8.png}}
%         \vspace{-0.05in}
%         \caption{拓扑结构复杂的殷集交并}
%         \vspace{-0.1in}
%     \end{figure}
%     \begin{figure}[!htb]
%         \centering
%         \subfigure
%         {\includegraphics[width=0.27\textwidth]{fig/s10.png}} \quad
%         \subfigure
%         {\includegraphics[width=0.3\textwidth]{fig/s9.png}} \quad
%         \subfigure
%         {\includegraphics[width=0.27\textwidth]{fig/s11.png}}
%         \vspace{-0.05in}
%         \caption{几何特征复杂的殷集求交}
%         %   \vspace{0.2in}
%     \end{figure}
% \end{frame}

% \begin{frame}
%     \frametitle{论文在投}
%     \centering
%     \includegraphics[height = \textheight]{fig/SIAM_Review.png}
% \end{frame}

% \section{博士阶段的研究计划}
% \subsection{流相的拓扑变化}
% \begin{frame}
%     \frametitle{博士研究计划}
%     \begin{center}
%         % \includegraphics[width = 0.7\textwidth]{fig/fig/Mars.gif};
%         \animategraphics[autoplay,
%             loop,
%             width=.5\textwidth]{5}{fig/gif/1/vortex80N64step00}{0}{62}
%     \end{center}
%     \begin{itemize}
%         \item 现有方法不能精确捕捉和刻画拓扑变化.
%         \item 结合殷集对流体建模,在MARS方法(Zhang, \tnr{ SISC}\tnrc{, \ 2018)}上
%               新增对拓扑变化的处理,并且高精度捕捉拓扑变化的时间点和位置.
%         \item 这个研究是三维殷集在军工项目中的重要应用,是项目的核心部分.
%     \end{itemize}
% \end{frame}

% \section*{}
% \begin{frame}
%     \centering\huge
%     \textcolor{red}{请各位老师批评指正!}
% \end{frame}
\end{document}