\documentclass[a4paper]{book}

\usepackage{geometry}
% make full use of A4 papers
\geometry{margin=1.5cm, vmargin={0pt,1cm}}
\setlength{\topmargin}{-1cm}
\setlength{\paperheight}{29.7cm}
\setlength{\textheight}{25.1cm}

% auto adjust the marginals
\usepackage{marginfix}

\usepackage{amsfonts}
\usepackage{amsmath}
\usepackage{amssymb}
\usepackage{amsthm}
%\usepackage{CJKutf8}   % for Chinese characters
\usepackage{ctex}
\usepackage{enumerate}
\usepackage{graphicx}  % for figures
\usepackage{layout}
\usepackage{multicol}  % multiple columns to reduce number of pages
\usepackage{mathrsfs}  
\usepackage{fancyhdr}
\usepackage{subfigure}
\usepackage{tcolorbox}
\usepackage{tikz-cd}
\usepackage{listings}
\usepackage{xcolor} %代码高亮
\usepackage{braket}
\usepackage{algorithm} 
\usepackage{algorithmicx}  
\usepackage{algpseudocode}  
\usepackage{amsmath}  

\floatname{algorithm}{算法}  
\renewcommand{\algorithmicrequire}{\textbf{输入:}}  
\renewcommand{\algorithmicensure}{\textbf{输出:}}  
\renewcommand{\algorithmicrequire}{\textbf{Input : }}
\renewcommand{\algorithmicrequire}{\textbf{Precondition : }}
\renewcommand{\algorithmicensure}{\textbf{Output : }}
\renewcommand{\algorithmicensure}{\textbf{Postcondition : }}
%------------------
% common commands %
%------------------
% differentiation
\newcommand{\gen}[1]{\left\langle #1 \right\rangle}
\newcommand{\dif}{\mathrm{d}}
\newcommand{\difPx}[1]{\frac{\partial #1}{\partial x}}
\newcommand{\difPy}[1]{\frac{\partial #1}{\partial y}}
\newcommand{\Dim}{\mathrm{D}}
\newcommand{\avg}[1]{\left\langle #1 \right\rangle}
\newcommand{\sgn}{\mathrm{sgn}}
\newcommand{\Span}{\mathrm{span}}
\newcommand{\dom}{\mathrm{dom}}
\newcommand{\Arity}{\mathrm{arity}}
\newcommand{\Int}{\mathrm{Int}}
\newcommand{\Ext}{\mathrm{Ext}}
\newcommand{\Cl}{\mathrm{Cl}}
\newcommand{\Fr}{\mathrm{Fr}}
% group is generated by
\newcommand{\grb}[1]{\left\langle #1 \right\rangle}
% rank
\newcommand{\rank}{\mathrm{rank}}
\newcommand{\Iden}{\mathrm{Id}}

% this environment is for solutions of examples and exercises
\newenvironment{solution}%
{\noindent\textbf{Solution.}}%
{\qedhere}
% the following command is for disabling environments
%  so that their contents do not show up in the pdf.
\makeatletter
\newcommand{\voidenvironment}[1]{%
\expandafter\providecommand\csname env@#1@save@env\endcsname{}%
\expandafter\providecommand\csname env@#1@process\endcsname{}%
\@ifundefined{#1}{}{\RenewEnviron{#1}{}}%
}
\makeatother

%---------------------------------------------
% commands specifically for complex analysis %
%---------------------------------------------
% complex conjugate
\newcommand{\ccg}[1]{\overline{#1}}
% the imaginary unit
\newcommand{\ii}{\mathbf{i}}
%\newcommand{\ii}{\boldsymbol{i}}
% the real part
\newcommand{\Rez}{\mathrm{Re}\,}
% the imaginary part
\newcommand{\Imz}{\mathrm{Im}\,}
% punctured complex plane
\newcommand{\pcp}{\mathbb{C}^{\bullet}}
% the principle branch of the logarithm
\newcommand{\Log}{\mathrm{Log}}
% the principle value of a nonzero complex number
\newcommand{\Arg}{\mathrm{Arg}}
\newcommand{\Null}{\mathrm{null}}
\newcommand{\Range}{\mathrm{range}}
\newcommand{\Ker}{\mathrm{ker}}
\newcommand{\Iso}{\mathrm{Iso}}
\newcommand{\Aut}{\mathrm{Aut}}
\newcommand{\ord}{\mathrm{ord}}
\newcommand{\Res}{\mathrm{Res}}
%\newcommand{\GL2R}{\mathrm{GL}(2,\mathbb{R})}
\newcommand{\GL}{\mathrm{GL}}
\newcommand{\SL}{\mathrm{SL}}
\newcommand{\Dist}[2]{\left|{#1}-{#2}\right|}

\newcommand\tbbint{{-\mkern -16mu\int}}
\newcommand\tbint{{\mathchar '26\mkern -14mu\int}}
\newcommand\dbbint{{-\mkern -19mu\int}}
\newcommand\dbint{{\mathchar '26\mkern -18mu\int}}
\newcommand\bint{
{\mathchoice{\dbint}{\tbint}{\tbint}{\tbint}}
}
\newcommand\bbint{
{\mathchoice{\dbbint}{\tbbint}{\tbbint}{\tbbint}}
}





%----------------------------------------
% theorem and theorem-like environments %
%----------------------------------------
\numberwithin{equation}{chapter}
\theoremstyle{definition}

\newtheorem{thm}{Theorem}[chapter]
\newtheorem{axm}[thm]{Axiom}
\newtheorem{alg}[thm]{Algorithm}
\newtheorem{asm}[thm]{Assumption}
\newtheorem{defn}[thm]{Definition}
\newtheorem{prop}[thm]{Proposition}
\newtheorem{rul}[thm]{Rule}
\newtheorem{coro}[thm]{Corollary}
\newtheorem{lem}[thm]{Lemma}
\newtheorem{exm}{Example}[chapter]
\newtheorem{rem}{Remark}[chapter]
\newtheorem{exc}[exm]{Exercise}
\newtheorem{frm}[thm]{Formula}
\newtheorem{ntn}{Notation}

% for complying with the convention in the textbook
\newtheorem{rmk}[thm]{Remark}


%\lstset{
%	backgroundcolor=\color{red!50!green!50!blue!50},%代码块背景色为浅灰色
%	rulesepcolor= \color{gray}, %代码块边框颜色
%	breaklines=true,  %代码过长则换行
%	numbers=left, %行号在左侧显示
%	numberstyle= \small,%行号字体
%	keywordstyle= \color{blue},%关键字颜色
%	commentstyle=\color{gray}, %注释颜色
%	frame=shadowbox%用方框框住代码块
%}
\lstset{
columns=fixed,       
numbers=left,                                        % 在左侧显示行号
numberstyle=\tiny\color{gray},                       % 设定行号格式
frame=none,                                          % 不显示背景边框
backgroundcolor=\color[RGB]{245,245,244},            % 设定背景颜色
keywordstyle=\color[RGB]{40,40,255},                 % 设定关键字颜色
numberstyle=\footnotesize\color{darkgray},           
commentstyle=\it\color[RGB]{0,96,96},                % 设置代码注释的格式
stringstyle=\rmfamily\slshape\color[RGB]{128,0,0},   % 设置字符串格式
showstringspaces=false,                              % 不显示字符串中的空格
language=c++,                                        % 设置语言
}

%----------------------
% the end of preamble %
%----------------------

\begin{document}
\pagestyle{empty}
\pagenumbering{roman}

%\tableofcontents
%\clearpage

%\pagestyle{fancy}
%\fancyhead{}
%\lhead{Qinghai Zhang}
%\chead{Notes on Algebraic Topology}
%\rhead{Fall 2018}


% \setcounter{chapter}{0}
\pagenumbering{arabic}
% \setcounter{page}{0}

% --------------------------------------------------------
% uncomment the following to remove these environments 
%  to generate handouts for students.
% --------------------------------------------------------
% \begingroup
% \voidenvironment{rem}%
% \voidenvironment{proof}%
% \voidenvironment{solution}%


% each chapter is factored into a separate file.

\chapter{殷集的Betti数b1的计算}

\section{Preliminaries}

\begin{defn}
    一个有向的紧二流形是同胚于球面或者环面或环面的有限个的连通和.
\end{defn}

\begin{defn}
    黏合紧曲面$S$是一个二维连通紧流形或这种流形的一个商空间, 其商映射
    将多个与一维 CW 复形同胚的子集粘在一起; 将这个一维子集删除后
    该黏合紧曲面仍然是连通的.不妨将内部有界的黏合紧曲面称为正向黏合紧曲面,
    无界的内部称为负向黏合紧曲面,使用$-S$表示改变黏合紧曲面的方向.
\end{defn}

\begin{defn}
    每个殷集 $\partial \mathcal{Y} \not= \emptyset, \mathbb{R}^3$ 可被唯一地表示为
    \begin{equation}
        \mathcal{Y} = \cup^{\bot\bot}_j \cap_i \text{int}(S_{j,i}),
    \end{equation}
    其中 $j$ 是 $\mathcal{Y}$ 的连通分量的下标, $i$ 是表示每个连通分量
    的边界的黏合紧曲面集合的下标, $S_{j,i}$ 是两两几乎不交的有向黏合紧曲面,
    $\forall j, \{{S_{j,i}} | i \in \mathbf{N}\}$中有一个或没有正向黏合紧曲面,有任意
    多个负向黏合紧曲面.

    并且殷集的Betti数$b_{0}$等于$j$的个数, $b_2$等于负向黏合紧曲面的个数.
\end{defn}

\begin{thm}
    欧拉示性数是一个拓扑不变量,对任意CW复形,欧拉示性数可通过交替求和
    \begin{equation}
        \chi = k_0 - k_1 + k_2 - k_3 + \cdots 
    \end{equation} 
    定义, 其中$k_n$代表在复形中$n$维元胞的数量.
    
    并且,对任意拓扑空间,定义第$n$个Betti数$b_n$.欧拉示性数可被定义
    为Betti数的交替求和
    \begin{equation}
        \chi = b_0 - b_1 + b_2 - b_3 + \cdots.
    \end{equation}
\end{thm}

\begin{thm}
    三维空间中,球$\mathtt{D}$的欧拉示性数为1,Betti数
    \begin{equation*}
        b_k = \left\{ 
        \begin{aligned}
            &1 \qquad \text{if } k = 0 \\
            &0 \qquad \text{otherwise },
        \end{aligned}
        \right.
    \end{equation*}
    
    球面$\mathtt{S}$的欧拉示性数为2,Betti数
    \begin{equation*}
        b_k = \left\{ 
        \begin{aligned}
            &1 \qquad \text{if } k = 0,2 \\
            &0 \qquad \text{otherwise }.
        \end{aligned}
        \right.
    \end{equation*}
\end{thm}

\begin{thm}
    三维空间中,实心环$\mathtt{ST}$的欧拉示性数为0,Betti数
    \begin{equation*}
        b_k = \left\{ 
        \begin{aligned}
            &1 \qquad \text{if } k = 0,1 \\
            &0 \qquad \text{otherwise },
        \end{aligned}
        \right.
    \end{equation*}
    
    环面$\mathtt{T}$的欧拉示性数为0,Betti数
    \begin{equation*}
        b_k = \left\{ 
        \begin{aligned}
            &1 \qquad \text{if } k = 0,2 \\
            &2 \qquad \text{if } k = 1 \\
            &0 \qquad \text{otherwise }.
        \end{aligned}
        \right.
    \end{equation*}
\end{thm}

\begin{thm}
    三维空间中, $k$个环面连通和的内部(不妨命名为$k$-实心环$\mathtt{ST^k}$)的欧拉示性数为$1 - k$,Betti数
    \begin{equation*}
        b_k = \left\{ 
        \begin{aligned}
            &1 \qquad \text{if } k = 0 \\
            &k \qquad \text{if } k = 1 \\
            &0 \qquad \text{otherwise },
        \end{aligned}
        \right.
    \end{equation*}
    
    $k$个环面的连通和(不妨命名为$k$-环面$\mathtt{T^k}$)的欧拉示性数为$2 - 2k$,Betti数
    \begin{equation*}
        b_k = \left\{ 
        \begin{aligned}
            &1 \qquad \text{if } k = 0,2 \\
            &2k \qquad \text{if } k = 1 \\
            &0 \qquad \text{otherwise }.
        \end{aligned}
        \right.
    \end{equation*}
\end{thm}

Theorem 1.7. 可通过计算连通两个实心环或环面时的欧拉示性数的变化,并且结合显然的$b_0 = 1,
b_2 = 0 \text{ or } 1$得到$b_1$的值.

\begin{thm}
    对于一个连通区域的边界$S$,和被$\overline{\text{int}(S)}$包含的负向黏合紧曲面$S_{inner}$.
    % 若$S \cap S_{inner}$几乎为空,
    则欧拉示性数
    \begin{equation}
        \chi_{\text{int}(S) \cap \text{int}(S_{inner})} = \chi_{\text{int}(S)}
        - (\chi_{\text{int}(-S_{inner})} - \chi_{S_{inner}}).
    \end{equation}
\end{thm}

\begin{proof}
    将$\overline{\text{int}(S)}$和$\overline{\text{int}(-S_{inner})}$(有界区域)$, S_{inner}$切分为元胞的集合
    $\varGamma_{\text{int}(S) }, 
    \varGamma_{\text{int}(-S_{inner})}, \varGamma_{S_{inner}}$.

    因为$ S_{inner} \subset \overline{\text{int}(-S_{inner})} \subset \overline{\text{int}(S)}$,不妨令
    \begin{align}
        &\varGamma_{\text{int}(-S_{inner})}  \subset \varGamma_{\text{int}(S)} \\
        &\varGamma_{S_{inner}} \subset \varGamma_{\text{int}(-S_{inner})} .
    \end{align}
    所以
    \begin{equation}
        \begin{aligned}
            \chi_{\text{int}(S) \cap \text{int}(S_{inner})} &= 
            \left| \varGamma_{\text{int}(S)} \right| - (\left|  \varGamma_{\text{int}(-S_{inner})} \right| 
            -  \left| \varGamma_{S_{inner}} \right|) \\
            &=  \chi_{\text{int}(S)} - (\chi_{\text{int}(-S_{inner})} - \chi_{S_{inner}}).
        \end{aligned}
    \end{equation}
\end{proof}

\begin{thm}
    令点$p \in \partial \mathcal{Y}$是一个殷集$\mathcal{Y} \subset \mathbb{R}^3$的边界
    上的点,对与任何充分小的邻域$N(p)$,我们有
    \begin{enumerate}[(a)]
        \item $\partial \mathcal{Y} \cap N(p)$是有限个广义圆盘的并集.
        \item (a)中广义圆盘两两之间的交除了点$p$以外是有限个两两之间交且仅交
        于点$p$的广义半径的并集.
        \item  $\partial \mathcal{Y} \cap N(p)$包含互相之间不交的规则开集;对于两个共享
        同一片广义扇形作为它们的边界的开集,其中一个是殷集$\mathcal{Y}$的子集,
        另一个是殷集补集$\mathcal{Y}^\bot $的子集.
    \end{enumerate}
\end{thm}

\begin{thm}
    对于殷集边界上的非流形点$p \in \partial \mathcal{Y}$,令$N(p)$为点$p$的的邻域
    当邻域充分小时,$\partial \mathcal{Y} \cap N(p)$可以被分割为有限个广义圆盘和
    折叠圆盘的并集.
    
    在$\partial \mathcal{Y} \cap N(p)$上的非流形点构成的广义半径上将
    广义扇形按好配对组合并与其他广义扇形分离,
    可以消除$\partial \mathcal{Y} \cap N(p)$内的非流形点.
\end{thm}

\begin{thm}
    黏合紧曲面$S$的欧拉示性数可如下计算
    
    $S$通过三角形表示,定义三角形顶点$p$和以$p$为顶点的三角形集合$\text{St}(p)$,
    将$\text{St}(p)$切分为多个集合$ M_p \subset \mathbb{N},
     \{\text{St}(p)_i \verb'|' i \in M_p\}$使得只有$\text{St}(p)_i$中的三角形
    可以沿以$p$为端点的边按好配对粘合起来(广义圆盘),则欧拉示性数满足
    \begin{equation*}
        \begin{aligned}
            \chi 
    &= \sum_p  \left| \{\text{St}(p)_i  \verb'|'  i \in M_p\} \right| - 
    \sum_p \sum_i \left| \text{St}(p)_i \right| / 6.
        \end{aligned}
    \end{equation*}
\end{thm}

\begin{proof}
    由上一条定理黏合紧曲面$S$可以通过好配对消除非流形点,所以
    $\{\text{St}(p)_i \verb'|' i \in M_p, p \in P_S\}$是流行上点的相邻三角形集合的集合,
    在邻域$N(p)$内构成没有非流形点的广义圆盘.
    每个三角形在
    $\{\text{St}(p)_i \verb'|' p \in P, i \in M_p\}$中出现三次(三个顶点),线段出现两次(一个三角形
    包含3条边,但是每条边被计算了6次(边同时在2个三角形中,且每个三角形被计算3次)),因此
    \begin{equation*}
        \begin{aligned}
            \chi &= k_0 - k_1 + k_2 \\
            &= \sum_p \left| \{\text{St}(p)_i \verb'|'  i \in M_p\} \right| - 
\sum_p \sum_i \left| \text{St}(p)_i \right| / 3 * 6 +
 \sum_p \sum_i \left| \text{St}(p)_i \right| / 3 \\
    &= \sum_p  \left| \{\text{St}(p)_i  \verb'|'  i \in M_p\} \right| - 
    \sum_p \sum_i \left| \text{St}(p)_i \right| / 6.
        \end{aligned}
    \end{equation*}
\end{proof}


\section{算法实现}

\subsection{输入}

\begin{itemize}
    \item 殷集$\mathcal{Y}$边界的黏合紧曲面表示$\{S_{j,i}\}$满足
    $\mathcal{Y} = \cup^{\bot\bot}_j \cap_i \text{int}(S_{j,i})$.
    
    \item 表示$\{{S_{j,i}}\}$的点$P$,线段,三角形.的集合.
    \item 对$P$中每个点$p$,点集的集合$\{\text{star}(p)_{j,i}\}$.
\end{itemize}

\subsection{precondition}

\begin{itemize}
    \item $\{{S_{j,i}} | i \in  \mathbf{N}\}$几乎只是殷集一个连通分量的边界(除点线).
    \item $\{{S_{j,i}} \}$是$\partial \mathcal{Y}$唯一表示.
    \item 对$P$中每个点$p$,在$S_{j,i}$上与$p$之间有线段的点集$\text{star}(p)_{j,i}$.
\end{itemize}

\subsection{输出}

\begin{itemize}
    \item Betti数$b_0, b_1, b_2$.
\end{itemize}

\subsection{计算过程}

\begin{enumerate}[a]
    \item 对$\mathcal{Y}$每个连通分量分别处理.若连通分量是无界的,
    $\{{S_{j,i}} | i \in \mathbf{N}\}$中没有正向的黏合紧曲面,不妨添加第一个黏合紧曲面
    是一个充分大的包含所有黏合紧曲面的球面.

    \item 对于$\{{S_{j,i}} | i \in \mathbf{N}\}$中唯一一个正向黏合紧曲面$S_{j,1}$,
    计算$\chi_{S_{j,1}}$得$S_{j,1} = \mathtt{S/T/T^k}$. 更新$b_{j,0}, b_{j,1}, b_{j,2}$为对应
    $\mathtt{D/ ST/ ST^k}$的值.若$\left|\{{S_{j,i}} | i \in \mathbf{N}\}\right| > 1$
    计算$\chi_{S_{j,2}}$,令
    \begin{align}
   &     \chi_{\text{int}(S_{j,1}) \cap \text{int}(S_{j,2})} = \chi_{\text{int}(S_{j,1})}
- (\chi_{\text{int}(-S_{j,2})} - \chi_{S_{j,2}}), \\
 &b_{j,2} = b_{j,2} + 1, \\
  &b_{j,1} =b_{j,0} + b_{j,2}
     - \chi_{\text{int}(S_{j,1}) \cap \text{int}(S_{j,2})}.
    \end{align}
    递归可得$\cap_i \text{int}(S_{j,i})$的$\chi_{\cap_i \text{int}(S_{j,i})} , b_{j, k}(k = 0,1,2)$.

    \item 选定黏合紧曲面$S_{j,i}$,插入$S_{j,i}$上的点$p_0$到$setp$内,直到出现点
    $p_1 \in setp$, 在三维坐标上$p_1 == p_0$.判断$p_0,p_1$是从
    $\cap_i \text{int}(S_{j,i})$内部粘合或从外部,
    如果是从内部粘合跳到(d),
    \begin{equation}
        \chi_{\cap_i \text{int}(S_{j,i})} = (k_0 - 1) - k_1 + k_2 =  
       \chi_{\cap_i \text{int}(S_{j,i})}  -1.
    \end{equation}
    定义$A := \{p_2\ \left|\  p_2 \in \text{star}(p_1)_{j,i}\ \verb'&&'
    \ p_0 \in \text{star}(p_2)_{j,i}\right. \}$.则
    \begin{equation}
       \chi_{\cap_i \text{int}(S_{j,i})} = k_0 - (k_1 - \left| A \right| )+ k_2 
        =\chi_{\cap_i \text{int}(S_{j,i})} +\left| A \right|.
    \end{equation}

    \item 
    $\text{star}(p_1)_{j,i} = \text{star}(p_1)_{j,i} \cup \text{star}(p_0)_{j,i}$,
    $\forall p_3 \in \text{star}(p_0)_{j,i}$, $p_1$替换$\text{star}(p_3)_{j,i}$中的$p_0$.
    跳到(c)直到$S_{j,i}$上所有点都插入$setp$中.
    
    清空$setp$,选取新的$i$跳到(c), $\{{S_{j,i}} | i \in \mathbf{N}\}$都处理后进行下一步.

    \item 更新$\cap_i \text{int}(S_{j,i})$的Betti数
    \begin{equation}
        b_{j,1} = b_{j,0} + b_{j,2} -\chi_{\cap_i \text{int}(S_{j,i})}.
    \end{equation}
    
    \item 选取另一个连通分量对应的$j$回到第(a)步, 直到所有连通分量都处理后
    \begin{align}
        &b_{0} = \sum_j b_{j,0}, \\
        &b_{1} = \sum_j b_{j,1}, \\
        &b_{2} = \sum_j b_{j,2}.
    \end{align}
\end{enumerate}

\begin{proof}
    第(a)步中每个连通分量单独处理,因为precondition第一条中
    $\{{S_{j,i}} | i \in \mathbf{N}\}$几乎只是一个连通分量的边界
    .并且将殷集无界区域使用充分大的球体包含不会影响
    殷集上的同调群.
    
    (b)中,使用定理1.11计算单个黏合紧曲面的欧拉示性数.
    % 可以以三角形为单元按好配对沿边粘合并且记录粘合后的
    % 欧拉示性数,将曲面上所有三角形粘合后得曲面的欧拉示性数(例如粘合四面体边界
    % 的四个三角形,欧拉示性数变化过程为$0\rightarrow 1 \rightarrow 1 \rightarrow 1
    % \rightarrow 2$).

    通过定理1.2和定理1.5,1.6,1.7可知外边界内部的Betti数.
    与内表面的内部求交结果的欧拉示性数结果(1.8)由定理1.8得到,由定义1.3
    连通分量不变,洞数量增加1(1.9).定理1.4有$b_{j,1}$(1.10)
    
    (c,d)中处理会改变欧拉示性数的粘合,
    通过粘合点相邻三角形集合之间的相对位置和方向判断两个点是否是从连通分量外部粘合.

    当从$\cap_i \text{int}(S_{j,i})$外部粘合时,(1.11)是点的减少,
    (1,12)是边的减少.显然从外部沿着1-CW复形粘合时3-cell三角形和4-cell四面体数量不会变化.
    
    但当从$\cap_i \text{int}(S_{j,i})$内部粘合,粘合过程将一个同胚于球的元胞集合的并压缩成圆盘,
    因为球和圆盘的欧拉示性数,Betti数都一致,所以$\chi_{\cap_i \text{int}(S_{j,i})},b_{j,k}$不变.

    (e)如果粘合过程中产生了连通分量个数变化和洞的个数变化,
    和定义1.3连通分量个数等于正向黏合紧曲面数量(对于无界殷集
    添加充分大的球面后),洞的个数等于负向黏合紧曲面数量
    说明产生了新的黏合紧曲面,与$\{S_{j,i}\}$是殷集边界$\partial \mathcal{Y}$的唯一表示
    矛盾,因此$b_{j,0}, b_{j,2}$不变,结合三维空间中更高次Betti数是0和
    定理1.4有$b_{j,1}$(1.13).
    
    (f)因为各个连通分量之间的交为空,所以殷集Betti数是各个连通
    分量Betti数的和(1.14-16).
\end{proof}





\end{document}
