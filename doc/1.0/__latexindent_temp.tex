\documentclass[UTF8]{ctexart}
\usepackage{amsmath,amssymb}
\usepackage{mathrsfs}
\usepackage{indentfirst} 
\usepackage{graphicx, subfig}
\setlength{\parindent}{2em}
\begin{document}
% \chapter{类成员简介}
	\section{Yinset}
	\subsection{成员数据}
	\textbf{vector<GluingClosedSurface> vecGCS}
	
	这里存放了构成殷集边界的有向黏合紧曲面
	
	\textbf{vector<HasseNode> Hasse}
	
	存放了有向黏合紧曲面之间的包含关系
	\subsection{成员函数}
	\textbf{Yinset meet(const Yinset\&) const}
	
	实现了两个殷集的求交
	
	\textbf{Yinset join(const Yinset\&) const}
	
	实现了两个殷集的求并
	
	\textbf{Yinset complement() const}
	
	实现了殷集求补集
	
	\textbf{buildHasse()}
	
	计算黏合紧曲面的包含关系,输出Hasse图
	
	
	\section{GluingClosedSurface}
	表示一个黏合紧曲面
	
	\subsection{成员数据}
	\textbf{vector<Triangle> vecTriangle}
	
	存放了黏合紧曲面的三角剖分
	
	\textbf{bool orientation}
	
	存放了黏合紧曲面的方向
	
	\section{SurfacePatch}
	表示切割后的曲面片
	
	\subsection{成员数据}
	\textbf{vector<Triangle> vecTriangle}
	
	存放了曲面片的三角剖分
	
	\textbf{vector<pair<Segment> boundary}
	
	存放了曲面片的边界
	
	\subsection{成员函数}
	\textbf{reverse()}
	
	将曲面片反向,也就是将所有三角形的顶点顺序取反
	
	
	\section{PrePaste}
	实现了将曲面沿闭合交线切割成曲面片的过程
	
	
	\subsection{成员数据}
	\textbf{vector<GluingClosedSurface> vecGCS}
	
	存放了不需要进行切割的黏合紧曲面
	
	\textbf{vector<SurfacePatch> vecSP}
	
	存放了切割以后得到的曲面片
	
	\subsection{成员函数}
	\textbf{operator()(const vector<Triangle>\&)}
	
	将一个殷集中所有三角形放在一起作为输入,将这些三角形黏合起来,直到遇到边界,这个过程等效于将黏合紧曲面沿交线进行切割。
	
	
		\section{Paste}
	实现了将曲面片沿边界黏合成黏合紧曲面的过程
	
	
	
	\subsection{成员函数}
	\textbf{vector<GluingClosedSurface> operator()(const vector<SurfacePatch>\&)}
	
	将输入的曲面片沿边界黏合成黏合紧曲面并输出
	
	\section{Locate}
	
	\subsection{成员函数}
	\textbf{bool operator()(const Point\&, const GluingClosedSurface\&)}
	
	判断一个点是否在一个黏合紧曲面的有界补集内部
	
	\section{TriangleIntersect}
	
	\subsection{成员数据}
	vector<pair<vector<Segment>,
	vector<vector<Triangle>::iterator>>> :
	resultA, reasultB
	
	存放了两个殷集的所有三角形之间相交的信息和重合的信息
	
	\subsection{成员函数}
	\textbf{operator()(const Triangle\&, const
		Triangle\&)}
	
	实现两个三角形的求交
	
	\textbf{vector<Triangle> collapse()}
	将所有三角形根据相交信息进行三角剖分
	
	\section{Triangulate}
	
	\subsection{成员函数}
	
	\textbf{bool operator()(const Triangle\&, const vector<Segment>\&)}
	将输入的三角形根据交线进行三角剖分
	
	
		\section{Triangle}
	实现了三角剖分所需的三角形
	
	
	\subsection{成员数据}
	\textbf{vector<Point> vecPoint}
	
	存放了三角形三个顶点,顶点顺序与定向有关
	
	% \textbf{vector<Edge> Edges}
	% 存放了三角形的三条边
\textbf{pairt<int,int> InFace}
	记录在哪一个曲面中
	
	\subsection{成员函数}
	\textbf{Triangle<2> project(int n)}
	
	将三维空间三角形投影到某个坐标平面
	
	\textbf{intersect(const Line\&)}
	
	实现空间中三角形与一条直线求交
	
	\textbf{intersectCoplane(const Line<2>\&)}
	实现平面中三角形和直线求交
	
	\textbf{Triangle reverse()}
	将三角形顶点顺序反向
	
	
	\section{Plane}
	表示三角形所在平面
	
	
	\subsection{成员数据}
	\textbf{Real para[Dim+1]}
	
	存放了平面方程的四个参数
	
	
	
	\subsection{成员函数}
	\textbf{ Real angle(const Plane\&)}
	
	求两个平面的夹角
	
	\textbf{Line intersect(const Plane\&}
	
	实现两个平面求交,输出交的直线
	
	
	\section{Line}
	表示一条直线
	
	
	\subsection{成员数据}
	\textbf{Point fixPoint}
	
	存放了直线上一点坐标
	
	\textbf{Vec dirct}
	
	存放了直线的方向向量
	
	
	\subsection{成员函数}
	\textbf{ Line<2> project(int n)}
	
	将空间中直线投影到某个坐标平面

	
	\section{Edge}
	表示一条线段
	
	
	\subsection{成员数据}
	\textbf{Point endPoint[2]}
	
	表示线段的两个端点
	
	
	
	\subsection{成员函数}
	\textbf{Edge<2> project(int n) }
	
	将空间中线段投影到某个坐标平面
	
	
	\section{Segment}
	表示一条交线
	
	
	\subsection{成员数据}
	\textbf{Point endPoint[2]}
	
	表示交线的两个端点
	
	\textbf{vector<Triangle>}
	
	存放了交线对应的两个三角形
	
	
	
	\section{Point}
	表示空间中一个点
	
	
	\subsection{成员数据}
	\textbf{Real coord[Dim]}
	
	表示点的坐标
	
	
	\section{Vec}
	表示一个向量
	
	
	\subsection{成员数据}
	\textbf{Real p[Dim]}
	
	表示向量的各个分量
	
	
	
	\subsection{成员函数}
	\textbf{Real dot(const Vec\&) }
	
	实现向量点乘

	\textbf{Real cross(const Vec\&) }

	实现向量叉乘	
	
	
	
\end{document}
